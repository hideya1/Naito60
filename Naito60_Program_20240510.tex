\documentclass[12pt,landscape]{jarticle}
\usepackage{amsmath,amssymb,amsthm,ascmac,amsfonts,mathrsfs}
\usepackage{enumerate,xhfill,colortbl}

\allowdisplaybreaks

\setlength{\textwidth}{257mm}
\setlength{\oddsidemargin}{-5.5mm}
\setlength{\textheight}{160mm}
\setlength{\topmargin}{0mm}
\setlength{\headheight}{0mm}
\setlength{\headsep}{0mm}

\pagestyle{empty}

\begin{document}

\begin{table}[h]
\centering
\begin{tabular}{|c||c|c|c|c|c|}
\hline
 & June 24th (Mon) & June 25th (Tue) & June 26th (Wed) & June 27th (Thu) & June 28th (Fri) \\
\hline \hline
9:30--10:30 & \cellcolor{gray} & Noriyuki Abe & Toshiki Nakashima & Hideya Watanabe & Hiroyuki Ochiai \\
\hline
10:50--11:50 & \cellcolor{gray} & Hiraku Nakajima & Masato Okado & Akihito Hora & Yoshihisa Saito \\
\hline \hline
13:20--14:20 & Daniel Orr & Masahiko Miyamoto & \cellcolor{yellow} & Naoki Fujita & \cellcolor{gray} \\
\hline
14:40--15:40 & Soichi Okada & Susumu Ariki & \cellcolor{yellow}Free discussion & Cristian Lenart & \cellcolor{gray} \\
\hline
16:00--17:00 & Jae-Hoon Kwon & Toshiaki Shoji & \cellcolor{yellow} & Satoshi Naito & \cellcolor{gray} \\
\hline
\end{tabular}
\end{table}

%%%%%%%%%%

\begin{center}
{\bf\Large June 24th (Mon)}
\end{center}

\noindent
%%%%
{\bf\large Daniel Orr (Virginia Tech)} \\
{\bf Functions on the Iwahori group} \\[1mm]
Abstract. 
In joint work with E. Feigin, A. Khoroshkin, and I. Makedonskyi, we study the bimodule structure of the algebra of functions on the Iwahori (i.e., affine Borel) subgroup of an affine Kac--Moody group. We prove an affine version of a finite-dimensional result of van der Kallen, namely: the algebra of functions on the Iwahori group admits a filtration with explicitly determined subquotients given by tensor products of standard and costandard objects in a suitable category of modules over its Lie algebra. Moreover, we show that these categories are stratified (i.e., generalized highest weight). At the level of graded characters, these considerations reveal new identities for specialized nonsymmetric Macdonald polynomials.
\\[5mm]
%%%%
{\bf\large Soichi Okada (Nagoya University)} \\ 
{\bf Plane partition enumeration via classical group characters} \\[1mm]
Abstract. 
A plane partition is a two-dimensional array of non-negative integers with weakly decreasing rows and columns and finitely many nonzero entries. The study of plane partitions goes back to MacMahon around 1900. There are several powerful methods to enumerate plane partitions and related objects, and one of them is the use of classical group characters such as Schur functions. In this talk, we will explain how we can utilize identities for classical group characters in enumeration problems of shifted plane partitions with some constraints.
\\[5mm]
%%%%
{\bf\large Jae-Hoon Kwon (Seoul National University)} \\
{\bf Extremal weight modules and quantum Verma Howe duality} \\[1mm]
Abstract. 
In this talk, we give a representation-theoretic interpretation of the generalized Cauchy identity, a generating function of the commutation relations of two kinds of Schur operators on the set of partitions.  We construct a Fock space of infinite level, which is a representation of the tensor product of a parabolic $q$-boson algebra and the quantum group of type $A_{+\infty}$, and then show that it has a multiplicity-free decomposition into its irreducible representations. A non-commutative character of this decomposition recovers the generalized Cauchy identity.
This is an on-going joint work with Soo-Hong Lee.
\\
\hrulefill

%%%%%%%%%%

\begin{center}
{\bf\Large June 25th (Tue)}
\end{center}

\noindent
%%%%
{\bf\large Noriyuki Abe (University of Tokyo)} \\
{\bf On Hecke categories} \\[1mm]
Abstract. 
Hecke categories play important role in representation theory, especially in representation theory of algebraic representations of reductive groups. There are several reincarnations of Hecke categories. In this talk we will discuss one of this based on bimodules. This is a generalization of the category of Soergel bimodules.
\\[5mm]
%%%%
{\bf\large Hiraku Nakajima (Kavli IPMU, the University of Tokyo)} \\
{\bf Geometry of nilpotent orbits for classical groups} \\[1mm]
Abstract. 
I have recently study Coulomb branches of orthosymplectic quiver gauge theories with Finkelberg and Hanany. They are closely related to nilpotent orbits for $SO$ and $Sp$, and their covers. I will report on this study with focus on the geometry of nilpotent orbits.
\\[5mm]
%%%%
{\bf\large Masahiko Miyamoto (University of Tsukuba)} \\
{\bf Application of Borcherds's Lie algebra} \\[1mm]
Abstract. 
One of main problems of VOAs is the uniqueness conjecture of VOAs of moonshine type. 
Namely, we want to prove that if $V$ is a VOA of central charge $24$ with $\mathrm{ch}_V(\tau)=J(\tau)$, then $V \cong V^{\natural}$. 
Borcherds has proved that the Borcherds's Lie algebra $B(V)$ is a generalized Kac--Moody algebra and $B(V) \cong B(V^{\natural})$. 
In this talk, we will show a few application of $B(V)$. 
For example, we will show that $V$ is $C_2$-cofinite and the space of $1$-point functions associated to $V$ coincides with the space associated to $V^{\natural}$. 
\\[5mm]
%%%%
{\bf\large Susumu Ariki} \\
{\bf A new class of symmetric algebras to study} \\[1mm]
Abstract. 
Obtaining character formulas or decomposition numbers is a difficult problem. In the case of the group algebras of the symmetric group, we have seen a certain progress in conceptual understanding in the past decade, but if we allow taking an indecomposable direct summand in the algorithm, we already saw a similar level of understanding in a more 
simple framework as we find in Stewart Martin's old book: namely we do not think that the problem is solved. On the other hand, those group algebras belong to the wide class of cyclotomic quiver Hecke algebras and there are other interesting cellular algebras in the class. Obtaining decomposition numbers for those algebras is difficult again, but we may find a subclass, and we expect that we may tackle the difficult problem in the subclass. We claim that the class of cyclotomic cellular quiver Hecke algebras of tame representation type may serve as such a subclass. In this talk, I will explain why this class of algebras is suitable for the study in our context. 
\\[5mm]
%%%%
{\bf\large Toshiaki Shoji (Tongji University)} \\
{\bf Generalized Green functions of reductive groups} \\[1mm]
Abstract. 
Generalized Green functions are important objects for computing 
irreducible characters of finite reductive groups, introduced by Lusztig. 
It is known by Lusztig that the computation of generalized Green functions 
is reduced to the computation of certain functions $Y_i$, 
which have a simple expression, up to scalar. 
But the behavior of those scalars is very subtle. 
In this talk, I will discuss about the problem of determining those scalars.   
This is a joint work with Frank Luebeck. 
\\
\hrulefill

%%%%%%%%%%

\begin{center}
{\bf\Large June 26th (Wed)}
\end{center}

\noindent
%%%%
{\bf\large Toshiki Nakashima (Sophia University)} \\
{\bf Localized quantum unipotent coordinate category and cellular crystals} \\[1mm]
Abstract. 
For a monoidal category $\mathcal{T}$, 
if there exists a ``real commuting family 
$(C_i,R_{C_i},\phi_i)_{i \in I}$", 
we can define a localization $\widetilde{\mathcal{T}}$ of $\mathcal{T}$ 
by 
$(C_i,R_{C_i},\phi_i)_{i \in I}$. 
Let $R = R(\mathfrak{g})$ be the quiver Hecke algebra(=KLR algebra) associated with a symmetrizable Kac--Moody Lie algebra $\mathfrak{g}$ and $\mathscr{C}_w$ 
the subcategory of $R$-gmod(=the category of graded finite-dimensional $R$-modules) defined by using a Weyl group element $w$, 
which is a monoidal category with a real commuting family 
$(C_i,R_{C_i},\phi_i)_{i \in I}$.
Thus, we get its localization $\widetilde{\mathscr{C}_w}$, 
which is called a ``localized quantum unipotent coordinate category" 
associated with $w$. 
In the talk, we present that for a 
(semi-)simple $\mathfrak{g}$ and the longest element $w_0$, 
the family of self-dual simple modules in
$\widetilde{\mathscr{C}_{w_0}} = \widetilde{R\text{-gmod}}$ 
holds a crystal structure and is isomorphic to the cellular crystal
$\mathbb{B}_{i_1 \ldots i_N}$
where $i_1 \ldots i_N$ is the reduced word of $w_0$. 
Furthermore, if time permits,
the latest result will be introduced that for a general 
symmetrizable Kac--Moody Lie algebra $\mathfrak{g}$ 
and a general Weyl group element $w$, 
the family of self-dual simple modules in $\widetilde{\mathscr{C}_w}$ 
holds a crystal structure and is isomorphic to a cellular crystal
associated with $w$, which is a joint work with M. Kashiwara.
\\[5mm]
%%%%
{\bf\large Masato Okado (Osaka Metropolitan University)} \\ 
{\bf Kirillov--Reshetikhin crystals and combinatorial $K$-matrices} \\[1mm]
Abstract. 
Using the $\imath$crystal theory by Watanabe for $\imath$quantum groups of quasi-split type, one can investigate a combinatorial $K$-matrix, $q$ to $0$ limit of the quantum $K$-matrix satisfying the set-theoretical reflection equation. We give examples obtained from Kirillov--Reshetikhin crystals of type A corresponding to the first Dynkin node. This is a joint work of Hiroto Kusano and Hideya Watanabe.
\\
\hrulefill

%%%%%%%%%%

\begin{center}
{\bf\Large June 27th (Thu)}
\end{center}

\noindent
%%%%
{\bf\large Hideya Watanabe (Rikkyo University)} \\
{\bf Some non-Levi branching rules arising from quantum symmetric pairs} \\[1mm]
Abstract. 
Quantum symmetric pairs consist of a quantum group (quantized enveloping algebra) and its certain right coideal subalgebra ($\imath$quantum group).
As a special case, one can consider the quantum group $U_q(\mathfrak{gl}_n)$ and its coideal subalgebra $U^\imath(\mathfrak{so}_n)$ or $U^\imath(\mathfrak{sp}_n)$ (when $n$ is even).
The algebras $U^\imath(\mathfrak{so}_n)$ and $U^\imath(\mathfrak{sp}_n)$ are $q$-deformations of the universal enveloping algebras $U(\mathfrak{so}_n)$ and $U(\mathfrak{sp}_n)$, respectively, but they are NOT the quantum groups.
In particular, Kashiwara's crystal base theory and Kashiwara--Nakashima's tableau models are not applicable.
Recently, a new combinatorial representation theory ($\imath$canonical basis, $\imath$crystal base) for these algebras has been developed by the speaker to some extent.
In this talk, I will explain how one can use this theory to solve the non-Levi branching rules for $(\mathfrak{gl_n}, \mathfrak{so_n})$ and $(\mathfrak{gl_n}, \mathfrak{sp_n})$.
\\[5mm]
%%%%
{\bf\large Akihito Hora (Hokkaido University)} \\ 
{\bf Dynamical spin limit shape of Young diagrams associated with spin representations of symmetric groups} \\[1mm]
Abstract. 
The branching graph associated with an inductive system of finite groups has rich 
statistical structure. Standard limit theorems in probability theory can be applied to formulate and understand its asymptotic behavior. The concentration phenomenon in dual objects of groups is one of such an interesting problem in asymptotic representation theory widely studied so far. In the first part of the talk, I intend to give a modest survey on the limit shape problem of Young diagrams which has enjoyed plenty of studies by A. Vershik and many other authors. I will mainly focus on the approach from dynamical (i.e. macroscopic time-evolutional) points of view. The second part is devoted to the model produced by the branching rule for spin irreducible representations of symmetric groups. After observing a concentration phenomenon for shifted Young diagrams, I will describe evolution of the spin limit shape along a continuous time parameter by using devices in (free) probability theory. 
Since a spin version of the Jucys--Murphy element plays an important role in asymptotic 
analysis, I will remark some of its nice properties. 
\\[5mm]
%%%%
{\bf\large Naoki Fujita (Kumamoto University)} \\
{\bf Schubert calculus on Newton--Okounkov polytopes and semi-toric degenerations} \\[1mm]
Abstract. 
One approach to Schubert calculus is to realize Schubert classes as concrete combinatorial objects such as Schubert polynomials. Through an identification of the cohomology ring of the type $A$ full flag variety with the polytope ring of the Gelfand--Tsetlin polytopes, Kiritchenko--Smirnov--Timorin realized each Schubert class as a sum of reduced (dual) Kogan faces. In this talk, we discuss its generalization to Newton--Okounkov polytopes of flag varieties in general Lie type. Newton--Okounkov polytopes of flag varieties with desirable properties induce degenerations of Schubert varieties into unions of irreducible toric varieties, called semi-toric degenerations. Such semi-toric degenerations can be expected as combinatorial models of Schubert classes.
\\[5mm]
%%%%
{\bf\large Cristian Lenart (State University of New York at Albany)} \\
{\bf From representations of quantum affine algebras to the quantum $K$-theory of flag manifolds and combinatorics} \\[1mm]
Abstract. 
I present an overview of my long-term collaboration with Satoshi Naito and Daisuke Sagaki, focusing on the more recent results. I start by explaining a new twist of the classical connections between Lie algebra representations, flag manifolds $G/B$, and combinatorics. This relates certain modules over quantum affine algebras, the equivariant $K$-theory of semi-infinite flag manifolds, and the equivariant quantum $K$-theory of $G/B$. Based on these connections, my collaborators and I derived combinatorial multiplication formulas in $K$-theory and quantum $K$-theory, which are expressed in terms of the so-called quantum alcove model. These formulas led us to several applications, including solutions to longstanding conjectures. 
\\[5mm]
%%%%
%{\bf\large Takeshi Hirai (Kyoto University)} \\ 
%{\bf On projective (or spin) representations of finite groups} \\[1mm]
%Abstract. 
%Let $G$ be a finite group.  
%Schur multiplier $M(G)=H^2(G,{\mathbb C}^\times)$ of $G$ has 
%been studied heavily and the results are accumulated e.g., 
%in a book of Karpilovski.
%
%Our problem here is to discuss to give a complete set of 
%representatives (=CSR) of 
%irreducible projective representations (=IPRs) of $G$. 
%This means 
%\begin{enumerate}[(1)]
%\item
%construction of representation group 
%$R(G)$, a special central extension of $G$ by $M(G)$, i.e.,  
%$R(G)/M(G)\cong G$, \quad and  
%\item
%construction of a CSR of irreducible 
%linear representations (=ILRs) of $R(G)$.
%\end{enumerate}
%Here we are interesting in ordinary  
%(not special) small groups, but having ${\mathbb Z}_3$ 
%in direct product components in $M(G)$.
%\\[5mm]
%%%%
{\bf\large Satoshi Naito (Department of Mathematics, Tokyo Institute of Tecnology)} \\ 
{\bf Quantum Demazure operators in the Borel-type presentation of the equivariant quantum $K$-theory ring of flag manifolds of type $A$} \\[1mm]
Abstract. 
In this talk, 
we first give the Borel-type presentation of the 
torus-equivariant quantum $K$-theory ring 
$QK_H(Fl_{n+1})$ of the flag manifold $Fl_{n+1}$; 
this presentation is obtained as an application of 
the inverse Chevalley formula for the 
torus-equivariant $K$-group of the 
semi-infinite flag manifold associated to 
$SL_{n+1}(\mathbb{C})$. 
Then, by making use of the quantum Demazure operators 
$D_i^Q$, 
$1 \leq i \leq n$, 
we prove that quantum double Grothendieck polynomials, 
introduced by Lenart--Maeno, 
represent opposite Schubert classes in 
$QK_H(Fl_{n+1})$. 
Finally, we explain how the zeroth quantum Demazure operator $D_0^Q$ 
on a certain localization $QK_H(Fl_{n+1})_{\mathrm{loc}}$ of 
$QK_H(Fl_{n+1})$ will play an important role 
in the study of the Schubert classes in $QK_H(Fl_{n+1})$. 
This talk is mainly based on a joint work with D. Sagaki and T. Maeno.
\\
\hrulefill

%%%%%%%%%%

\begin{center}
{\bf\Large June 28th (Fri)}
\end{center}

\noindent
%%%%
{\bf\large Hiroyuki Ochiai (Institute of Mathematics for Industry, Kyushu University)} \\ 
{\bf A D-module approach to invariant distributions with finitely many orbits} \\[1mm]
Abstract. 
Tauchi provides an example illustrating the action of a real algebraic subgroup $H$ of $GL(2n,\mathbb{R})$ with finitely many orbits on $\mathbb{R}^{2n}$, 
while the dimension of the space of relative $H$-invariant distributions on $\mathbb{R}^{2n}$ is infinite. We offer a perspective on this example from the viewpoint of D-modules, where we explicitly determine the simple quotient regular holonomic D-modules and demonstrate that the distributions exhibit an enlarged symmetry.
\\[5mm]
%%%%
{\bf\large Yoshihisa Saito (Rikkyo University)} \\ 
{\bf Elliptic root systems and their applications} \\[1mm]
Abstract. 
In the middle of 1980's, motivated by study of singularity theories, K. Saito
introduced the notion of ``elliptic root systems". Roughly speaking, they are root
systems with two null directions. Furthermore, he had classified elliptic root systems
$R$ with a one-dimensional subspace of $G$ of two dimensional null directions, under the
assumption that the quotient affine root system $R/G$ is reduced. In our joint work
with A. Fialowski and K. Iohara, we take off the assumption above, and give the
classication of the pair $(R, G)$ with no assumption. In addition, we give an overview
of the theory of elliptic root systems in this talk, and certain applications of this theory 
are also discussed.

\end{document}